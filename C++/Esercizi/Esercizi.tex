\title{Esercizi C++}
\author{Giuseppe Facchi}
\date{}

\documentclass{article}
\usepackage{listings}
\usepackage{xcolor} % for setting colors
\usepackage[italian]{babel}
\usepackage{courier}
% set the default code style
\lstset{
    frame=tb, % draw a frame at the top and bottom of the code block
    tabsize=4, % tab space width
    showstringspaces=false, % don't mark spaces in strings
    numbers=left, % display line numbers on the left
    commentstyle=\color{green}, % comment color
    keywordstyle=\color{blue}, % keyword color
    stringstyle=\color{red},
    basicstyle=\large\ttfamily,
    breaklines=true, 
    framextopmargin=50pt,
    frame=bottomline
}
\begin{document}

\maketitle
\tableofcontents
\newpage
\vspace*{\fill}

\section{Somma} Somma 10 numeri interi e comunica la somma dopo averli letti o quando viene inserito lo 0
\begin{lstlisting}[language=C++]
#include <iostream>
int main()
{
    //Risultato
    int somma = 0;

    //Variabile di input
    int input;

    //Contatore per ciclo
    int i = 0; 
    do {
        cout << "Inserisci numero" << i << ": "; 
        cin >> input;
        somma = somma + input;
        i++;
    } while(input != 0 && i < 10);
    
    return 0;
}
\end{lstlisting}
\subsection{Prodotto} Moltiplica 10 numeri interi e comunica la somma dopo averli letti o quando viene inserito lo 0
\pagebreak
\section{Numeri Pari} Inserendo 10 numeri interi, calcola quanti numeri pari sono stati inseriti
\begin{lstlisting}[language=C++]
#include <iostream>
int main()
{
    //Prendo l'input dall'utente
    int interi[10];
    int i = 0;
    while(i < 10){
        cout << "Inserisci num " << i << ": ";
        cin >> interi[i];
        i++;
    }

    //Calcolo Numeri Pari
    int j = 0;
    int numPari = 0;
    while(j < 10){
        if(interi[j] % 2 == 0) {
            numPari = numPari + 1;
        }
        j++;
    }
    cout << "Inseriti " << numPari << " pari";
    return 0;
}
\end{lstlisting}
\subsection{Numeri Pari} Inserendo 10 numeri interi, stampare tutti i numeri pari
\subsection{Numeri Pari} Inserendo 10 numeri interi, stampare tutti i numeri in posizione pari
\pagebreak
\section{Elevamento a Potenza} Eleva un numero intero per un esponente inserito da tastiera

\begin{lstlisting}[language=C++]
#include <iostream>
int main()
{
    //Prendo l'input dall'utente
    int numDaElevare;
    cin >> numDaElevare;

    int esponente;
    cin >> esponente;

    //Calcolo
    int risultato;
    if(esponente == 0) {
        risultato = 1;
    } else {
        for(int i = 0; i < esponente; i++){
            risultato = risultato * risultato;
        }
    }

    //Stampo risultato
    cout << "Risultato: " << risultato;
    return 0;
}
\end{lstlisting}
\pagebreak
\section{Inversione Array} Carica da tastiera un array di 10 elementi e lo stampa invertendo i suoi elementi

\begin{lstlisting}[language=C++]
#include <iostream>
int main()
{
    int numeri[10];
    
    //Prendo l'input dall'utente
    for(int i = 0; i < 10; i++){
        cout << "\nInserisci n " << i << ": ";
        cin >> numeri[i];
    }
    //Stampo al contrario
    for(int j = 10; j > 0; j--){
        cout << numeri[j] << "\n";
    }
    return 0;
}
\end{lstlisting}
\subsection{Copia Array} Carica da tastiera un array di 10 elementi e stampane uno contenente i valori dell'array stesso seguito dalla sua inversione
\pagebreak
\section{Conta tipi di input} Programma che legge 10 caratteri da tastiera e comunica quante a,b,c... sono state inserite.

\begin{lstlisting}[language=C++]
#include <iostream>
int main()
{
    char caratteri[10];
    
    //Prendo l'input dall'utente
    for(int i = 0; i < 10; i++){
        cout << "\nIns car " << i << ": ";
        cin >> caratteri[i];
    }
    //Conto a, b, c
    int a, b, c;
    for(int j = 0; j < 10; j++){
        switch(caratteri[j]){
            case 'a':
                a = a + 1;
                break;
            case 'b':
                b = b + 1;
                break;
            case 'c':
                c = c + 1;
                break;
        }
    }
    //Stampo i risultati
    cout << "Numero di a: " << a << "\n";
    cout << "Numero di b: " << b << "\n";
    cout << "Numero di c: " << c << "\n";
    return 0;
}
\end{lstlisting}
\subsection{Input non validi} Dato lo stesso esercizio invece di stampare il numero di a, b, c, stampa il numero di input non validi
\pagebreak
\section{Media di Tre Numeri} Dati tre numeri in input, calcolane la media

\begin{lstlisting}[language=C++]
#include <iostream>
int main() 
{ 
    //Prendo l'input dall'utente
    float v1, v2, v3;
    cout << "Inserisci i tre voti " << "\n";
    cin >> v1;
    cin >> v2;
    cin >> v3;   

    //Calcolo media
    float somma = 0;
    float media_voti = 0;
    somma = v1 + v2 + v3;  
    media_voti= somma / 3;

    //Stampo media
    cout << endl << " Media voti :" << media_voti; 
}
\end{lstlisting}
\vspace*{\fill}

\pagebreak
\section{Vettore Random} Crea un vettore dim 20 caricalo con numeri casuali 2-17, stampalo normale e poi stampalo a rovescia

\begin{lstlisting}[language=C++]
#include <iostream>

using namespace std;

int main()
{
    //Dichiarazione Vettore
    int interi[20];
    
    //Iterazione Vettore
    for (int i = 0; i < 20; i++){
        //rand() % 10 --> Numero random DA 0 A 9 
        //rand() % 10 + offset --> Numero random DA "offset" A "9 + offset" (es. offset = 10 cosa restituisce?)
        interi[i] = rand() % 16 + 2;
    }
    
    //Iterazione Vettore
    cout << "Valori vettore: ";
    for (int j = 0; j < 20; j++) {
        cout << interi[j] << " ";
    }
    cout << endl;
    cout << "Valori inversi vettore: ";
    //Iterazione Vettore a rovescio
    for (int k = 19; k >= 0; k--) {
        cout << interi[k] << " ";
    }
}
\end{lstlisting}
\vspace*{\fill}

\pagebreak
\section{Ordinamento vettore} Ordinamento vettore crescente

\begin{lstlisting}[language=C++]
#include<iostream>

using namespace std;

int main()
{
    int i, j, min, temp;
    int a[10];

    cout<<"Inserisci gli elementi:\n"; 
    for(i = 0; i < 10; i++){
        cin >> a[i];
    }

    for(i = 0; i < 10-1; i++)
    {
	    min = i;
        for(j = i + 1; j < 10; j++)
        	if (a[j] < a[min])
        	     min = j;

        temp = a[min];
        a[min] = a[i];
        a[i] = temp;
    }
    
    cout<<"Array ordinato con selection sort:";
    for(i = 0; i < 10; i++)
	cout << " " << a[i];
}
\end{lstlisting}
\vspace*{\fill}
\pagebreak
\section{Matrice Quadrata} Matrice quadrata di ordine 5

\begin{lstlisting}[language=C++]
#include<iostream>

using namespace std;

int main()
{
    int matriceQuadrata[5][5];

    for(int i = 0; i < 5; i++){
        for(int j = 0; j < 5; j++){
            if(i == j){
                matriceQuadrata[i][j] = 2;
            }
            else{
                matriceQuadrata[i][j] = 0;
            }
        }
    }

    //Stamparla (non ti serve)
    for (int i = 0; i < 5; i++) 
        for (int j = 0; j < 5; j++)           
            cout << matriceQuadrata[i][j] << " \n"[j == 5-1]; 
}
\end{lstlisting}
\vspace*{\fill}

\pagebreak
\section{Perimetro e Area maggiore} Dato lato di un quadrato e raggio di un cerchio dire quale dei due ha area maggiore e perimetro maggiore, attenzione manca qualcosa!

\begin{lstlisting}[language=C++]
#include<iostream>
using namespace std;
int main()
{
    float latoQuadrato, raggioCerchio;

    cout << "Inserisci lato quadrato" << endl;
    cin >> latoQuadrato;
    cout << "Inserisci raggio cerchio" << endl;
    cin >> raggioCerchio;

    float aQuadrato = latoQuadrato * latoQuadrato;
    float pQuadrato = latoQuadrato * 4;

    float aCerchio = 3.14 * raggioCerchio * raggioCerchio;
    float pCerchio = 2 * 3.14 * raggioCerchio;

    if(aQuadrato > aCerchio){
        cout << "Il quadrato ha A maggiore" << endl;
    }
    else {
        cout << "Il cerchio ha A maggiore" << endl;
    }
    cout << "aCerchio: " << aCerchio << " aQuadrato: " << aQuadrato << endl;
    
    if(pQuadrato > pCerchio){
        cout << "Il quadrato ha P maggiore" << endl;
    }
    else {
        cout << "Il cerchio ha P maggiore" << endl;
    }
    cout << "pCerchio: " << pCerchio << " pQuadrato: " << pQuadrato << endl;
}
\end{lstlisting}
\vspace*{\fill}

\pagebreak
\section{Vettore random e stampare prodotto} Crea un vettore di dim 10, caricalo casuale 6-12, stampalo e restituisci il prodotto dei suoi componenti

\begin{lstlisting}[language=C++]
#include <iostream>

using namespace std;

int main()
{
    int interi[10];
    
    for (int i = 0; i < 10; i++){
        //rand() % 10 --> Numero random DA 0 A 9 
        //rand() % 10 + offset --> Numero random DA "offset" A "9 + offset"
        interi[i] = rand() % 6 + 6;
    }
    cout << "Valori vettore: ";
    for (int j = 0; j < 10; j++) {
        cout << interi[j] << " ";
    }
    cout << endl;

    int prodotto = 1; //Perche 1?   
    for (int k = 0; k < 10; k++){
        prodotto = prodotto * interi[k];
    }
    cout << "Prodotto: " << prodotto;
}
\end{lstlisting}
\vspace*{\fill}

\pagebreak
\section{Numero intero positivo e successione} Chiedere num positivo e che visualizzi la squenza dei primi 5 numeri successivi al numero dato

\begin{lstlisting}[language=C++]
#include <iostream>

using namespace std;

int main()
{
    int n;
    do {
        cout << "Inserisci valore intero positivo" << endl;
        cin >> n;

        if(n < 0){
            cout << "Valore non valido" << endl;
        }
    } while (n < 0);

    for(int i = 1; i < 6; i++){
        cout << n + i << " "; 
    }
}
\end{lstlisting}
\vspace*{\fill}
\clearpage

\end{document}
 